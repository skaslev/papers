\documentclass[a4paper]{article}

\usepackage[english]{babel}
\usepackage[utf8x]{inputenc}
\usepackage{amsmath}
\usepackage{amssymb}
\usepackage{graphicx}
\usepackage{hyperref}
\usepackage[colorinlistoftodos]{todonotes}

\usepackage{fancyvrb}
\DefineVerbatimEnvironment{code}{Verbatim}{fontsize=\small}
\DefineVerbatimEnvironment{example}{Verbatim}{fontsize=\small}
\newcommand{\ignore}[1]{}

\title{Combinatorics of general balanced trees}
\author{Slavomir Kaslev}

\begin{document}
\maketitle

\section{Introduction}

\begin{code}
  data F a = F0 a | F1 (F (G a))
  data G a = G0 a | G1 a (G a)
\end{code}

The generating function $f(x)$ of general balanced trees satisfies the equations
\begin{align*}
  f(x) &= x + f(g(x)) \\
  g(x) &= x + x g(x)
\end{align*}

\begin{equation}
  f(x) = x + \epsilon f(g(x))\label{eq:eq1}
\end{equation}

$$f(x) = x + \epsilon g(x) + \epsilon^2 g(g(x)) + \epsilon^3 g(g(g(x))) + \dots$$

\begin{align*}
g_0(x) &= x\\
g_1(x) &= g(x)\\
g_n(x) &= g_{n-1}(g(x))
\end{align*}

\begin{equation}
f(x) = \sum_{n=0}^{\infty}{g_n(x)} \epsilon^n
\end{equation}


Since
\begin{equation}
  g(x) = \frac{x}{1-x}
\end{equation}
one can easy show by induction that $\forall n \in \mathbb{N}$
$$g_n(x) = \frac{x}{1 - n x}$$

Moreover $\forall n \in \mathbb{N}^+$
$$g_n(x) = \frac{g(n x)}{n} $$
and $\forall n,m \in \mathbb{N}^{+}$
$$g_{nm}(x) = \frac{g_n(m x)}{m} = \frac{g_m(n x)}{n}$$

$g_n(x)$ can be expanded in power series
$$g_n(x) = x + n x^2 + n^2 x^3 + \dots + n^{k-1}x^k + \dots$$
$\forall k \in \mathbb{N}^{+}$
$$[x^k] g_n(x) = n^{k-1}$$
$$[\epsilon^n] f(x) = g_n(x)$$
$$[x^k\epsilon^n] f(x) = n^{k-1}$$

The functions $g_n(x)$ form a comutative monoid under function composition
with identity element $g_0(x) = x$ since $\forall n, m, l\in \mathbb{N}$ we have
$$g_0 \circ g_n = g_n \circ g_0 = g_n$$
$$g_n \circ g_m = g_m \circ g_n = g_{n+m}$$
$$(g_n \circ g_m) \circ g_l = g_n \circ (g_m \circ g_l) = g_{n+m+l}$$

$$f(g(x)) = f(x) - x = \sum_{n=1}^{\infty} g_n(x)$$
$$f(g(x)) = \sum_{n=1}^{\infty} \frac{g(n x)}{n}$$
$$g(x) = \sum_{m=1}^{\infty} \mu(m) \frac{f(g(m x))}{m}$$
$$\frac{f(g(m x))}{m} = \sum_{n=1}^{\infty} \frac{g_n(m x)}{m} = \sum_{n=1}^{\infty} g_{n m}(x)$$
$$g(x) = \sum_{m=1}^{\infty} \mu(m) \sum_{n=1}^{\infty} g_{n m}(x)$$
Which can quickly be verified by looking at the coefficients in front $x^k$ for $k \in \mathbb{N}^+$
$$1 = \sum_{m=1}^{\infty} \mu(m) \sum_{n=1}^{\infty} n^{k-1} m^{k-1} = \zeta(1-k) \sum_{m=1}^{\infty} \mu(m) m^{k-1} = \frac{\zeta(1-k)}{\zeta(1-k)}$$

\end{document}
